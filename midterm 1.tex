%% LyX 2.3.5.2 created this file.  For more info, see http://www.lyx.org/.
%% Do not edit unless you really know what you are doing.
\documentclass[english]{article}
\usepackage[latin9]{inputenc}
\usepackage{babel}
\begin{document}
Yinfeng Cong

netid: yc957

CS 323 

1. (a) ii, iii

(b) iii

(c) i

(d) ii

2. (a)

The problem with the Vandermonde matrix is that when the dimensionality
of the matrix gets higher and higher, the problem will become more
and more ill-conditioned when solving linear equations, and the condition
number will become larger and larger. And it is much simpler for Newton
interpolation to calculate than the Vandermonde matrix method. For
an incremental problem, Newton interpolation is also quicker and simpler.

When computing higher-order interpolation, the error of Newton interpolation
is bigger than Lagrange interpolation

(b) NO, bisection search is like binary search in another class. Its
mission is to find a root of it, and it is recursive to find the value
between two numbers (a bigger one and a smaller one), and finally
find the value we want to get. So it is not a fixed point method.

(c) 1. because the basis matrix is a lower triangular, and it is much
easier to list the triangular matrix than using Vandermonde matrix.
And when we plug in the data and calculate the triangular matrix,
we can directly get the answer easily. (We can calculate the answer
easily, but for Vandermonde matrix it is much more complex)

2. Also triangular matrix is easier to understand and write, we can
devide it to lower triangular, x values, y values three parts, it
is clearly easy to understand what we have here. However, we need
to m{*}(n-1) numbers together in the matrix for Vandermonde matrix,
it is much harder to understand from so many numbers.

3. 

We know that 
\[
x_{k+1}=g(x_{k})h(x_{k})
\]

and we also know that $g(x_{k}),h(x_{k})$are both guaranteed to converge
in the same interval

so there is a number x = m that makes $g'(m)=0$ and $h'(m)=0$

From the hint we can get

\[
[g(m)h(m)]'=g(m)h'(m)+g'(m)h(m)=0+0=0
\]

so it also exhibit quadratic convergence

4.

Using Lagrange interpolation

\[
f(x)=\frac{(x+1)(x-2)(x-4)}{(-2+1)(-2-2)(-2-4)}*(-5)+\frac{(x+2)(x-2)(x-4)}{(-1+2)(-1-2)(-1-4)}*(-6)
\]

\[
+\frac{(x+2)(x+1)(x-4)}{(2+2)(2+1)(2-4)}*(-8)+\frac{(x+1)(x+2)(x-2)}{(4+1)(4+2)(4-2)}*(-4)=
\]

\[
\frac{3x^{3}}{40}+\frac{19x^{2}}{120}-\frac{21x}{20}-\frac{107}{15}
\]

5.

The iteration function g(x) for Newton\textquoteright s method is:

\[
g(x)=x-\frac{f(x)}{f'(x)}=x-\frac{x^{2}-5}{2x}=x-\frac{1}{2}x+\frac{5}{2x}
\]

\[
g'(x)=\frac{1}{2}-\frac{5}{2x^{2}}
\]

The condition \textbar g'(x)\textbar{} \textless{} 1 is equivalently
written as

\[
|\frac{1}{2}-\frac{5}{2x^{2}}|<1
\]

\[
\Longrightarrow-1<\frac{1}{2}-\frac{5}{2x^{2}}<1
\]

\[
\Longrightarrow-\frac{1}{2}<\frac{5}{2x^{2}}<\frac{3}{2}
\]
($\frac{5}{2x^{2}}is$ always bigger than $-\frac{1}{2}$, so we do
not need to calculate about it any more)

\[
\Longrightarrow\frac{5}{x^{2}}<3\Longrightarrow3x^{2}>5\Longrightarrow x^{2}>\frac{5}{3}
\]

From the problem we know that we have the initial guess $x_{0}\geq2$,
so $x^{2}$always $\geq4$

which is also true for all x \textgreater{} $\frac{5}{3}$, indicating
that g is a contraction in this case

To make $f(a)=0$, we need to make $\frac{1}{2}-\frac{5}{2x^{2}}=0\Longrightarrow x^{2}=5$(bigger
than 4 so it is ok) , then

\[
g'(a)=\frac{1}{2}-\frac{5}{2x^{2}}=\frac{1}{2}-\frac{5}{2*5}=0
\]

As a result, convergence has to be quadratic.
\end{document}
