%% LyX 2.3.5.2 created this file.  For more info, see http://www.lyx.org/.
%% Do not edit unless you really know what you are doing.
\documentclass[english]{article}
\usepackage[latin9]{inputenc}
\usepackage{babel}
\begin{document}
1. 

(a).

$||x||_{\infty}=max_{1\leq i\leq n}\left\{ |x_{i}|\right\} $

and $||x||_{1}=\sum_{i=1}^{n}|x_{i}|$

we know $max_{1\leq i\leq n}\left\{ |x_{i}|\right\} \leq\sum_{i=1}^{n}|x_{i}$

so we get $||x||_{\infty}\leq||x||_{1}$

$n||x||_{\infty}=n*max_{1\leq i\leq n}\left\{ |x_{i}|\right\} =\sum_{i=1}^{n}max_{1\leq i\leq n}\left\{ |x_{i}|\right\} $

and $\sum_{i=1}^{n}|x_{i}\leq\sum_{i=1}^{n}max_{1\leq i\leq n}\left\{ |x_{i}|\right\} $

so $||x||_{\infty}\leq||x||_{1}\leq n||x||_{\infty}$

It is also the same for $x_{2}$

$||x||_{2}=\sqrt{x_{1}^{2}+x_{2}^{2}+...(max_{1\leq i\leq n}\left\{ |x_{i}|\right\} )^{2}+...x_{n}^{2}}\geq\sqrt{(max_{1\leq i\leq n}\left\{ |x_{i}|\right\} )^{2}}=||x||_{\infty}$

$||x||_{2}=\sqrt{x_{1}^{2}+x_{2}^{2}+...(max_{1\leq i\leq n}\left\{ |x_{i}|\right\} )^{2}+...x_{n}^{2}}\leq\sqrt{\sum_{i=1}^{n}(max_{1\leq i\leq n}\left\{ |x_{i}|\right\} )^{2}}=\sqrt{n*(||x||_{\infty})^{2}}=\sqrt{n}||x||_{\infty}$

so$||x||_{\infty}\leq||x||_{2}\leq\sqrt{n}||x||_{\infty}$

(b).

$||M||_{a}=max_{||x||_{a}=1}||Mx||_{a}$

$||M||_{b}=max_{||x||_{b}=1}||Mx||_{b}$

Also from lecture we know that $||Mx||_{a}\leq||M||_{a}||x||_{a}$,
and $||Mx||_{b}\leq||x||_{b}$

Then

$||M||_{a}=max_{||x||_{a}=1}||Mx||_{a}\leq max_{||x||_{a}=1}||M||_{a}||x||_{a}\leq\frac{1}{c_{1}}max_{||x||_{b}=1}||M||_{b}||x||_{b}$
(From $c_{1}||x||_{a}\le||x||_{b}$, we can get this, and of course
$||M_{b}||\geq||M||_{a}$)

Then we can know $||M||_{a}\leq\frac{1}{c_{1}}||M||_{b}$

so there must be a positive constant d satisfy the equation

Also

$|M||_{b}=max_{||x||_{b}=1}||Mx||_{b}\leq max_{||x||_{b}=1}||M||_{b}||x||_{b}\leq\frac{1}{c_{2}}max_{||x||_{a}=1}||M||_{a}||x||_{a}$(From
$c_{2}||x||_{a}\geq||x||_{b}$, we can get this)

Then we can know $d_{1}||M||_{a}\leq||M||_{b}\leq d_{2}||M||_{a}$

2.

(a).

$1*1-(1+\varepsilon)(1-\varepsilon)=1-1+\varepsilon^{2}=\varepsilon^{2}$

(b).

To make determinant equals 0, we need to make $\varepsilon^{2}=0$,
which means that $\varepsilon=0$

(c).

A = LU = $\left[\begin{array}{cc}
l_{11} & 0\\
l_{21} & l_{22}
\end{array}\right]\left[\begin{array}{cc}
u_{11} & u_{12}\\
0 & u_{22}
\end{array}\right]$

$l_{11}*u_{11}+0*0=1$

$l_{11}*u_{12}+0*u_{22}=1+\varepsilon$

$l_{21}*u_{11}+l_{22}*0=1-\varepsilon$

$l_{21}*u_{12}+l_{22}*u_{22}=1$

$l_{11}=1,l_{21}=1-\varepsilon,l_{22}=1,u_{11}=1,u_{12}=1+\varepsilon,u_{22}=\varepsilon^{2}$

so $L=\left[\begin{array}{cc}
1 & 0\\
1-\varepsilon & 1
\end{array}\right],U=\left[\begin{array}{cc}
1 & 1+\varepsilon\\
0 & \varepsilon^{2}
\end{array}\right]$

(d).

to make it singular, we need to make sure \textbar u\textbar{} =
0

$|u|=\varepsilon^{2}-0*(1+\varepsilon)=\varepsilon^{2}$

$\varepsilon^{2}=0,\varepsilon=0$


\end{document}
